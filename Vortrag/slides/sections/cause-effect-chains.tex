\subsection{Use case 2: Constructing Cause-Effect-Chains}
%\begin{frame}{Use case 2: Finding the reason a program fails}
%	\begin{itemize}
%		\item The \ddp\ algorithm as described above can be used to find out which part of a certain input is responsible for crashing a program.
%		\item From a developers perspective it would be great to know \textit{why} this specific input induces failure.
%	\end{itemize}


%\end{frame}

\begin{frame}{Use case 2: Finding the reason a program fails}

	\begin{alertblock}{Difficulty: Causalities from the input change to the failure}
		A minimal change in the input has major consequences for the program execution. How can the debugger know which chain of effects is responsible for the eventual failure?
	\end{alertblock}

	\pause
	\bigskip
	
	\begin{exampleblock}{Idea: Delta Debugging on program states}
		Use delta debugging on program states of a known succeeding run $r_\cmark$ and a known failing run $r_\xmark$ of the program. For each analyzed pair of program states we would then know what current difference causes the failure down the line.

		\bigskip

		$\Rightarrow$ A \textit{Cause-Effect-Chain} can be constructed from those comparisons.
	\end{exampleblock}	

\end{frame}

\begin{frame}{Delta Debugging on program states}
	\begin{itemize}
		\item A program state is essentially a set of $(variable, value)$ pairs.
		\item A program state of $r_\cmark$ and one of $r_\xmark$ can have the following differences (the deltas): \begin{itemize}
			\item A variable only present in one state
			\item A different value of a variable present in both states
		\end{itemize}
		%\item With this the runs $r_\cmark$ and $r_\xmark$ can be compared in different significant locations. 
		\item We use delta debugging to find the current $(variable, value)$ pairs that are responsible for the failure of $r_\xmark$ further down the line.
	\end{itemize}

	%\bigskip

	%$\Rightarrow$ This allows us to build a \textit{Cause-Effect-Chain} that helps understand why the program fails.
\end{frame}

\newcommand{\cecarrow}{\to}

\begin{frame}{\textit{HOWCOME} on \texttt{fail.c}}
	

	
	\begin{columns}
		\begin{column}{0.4\textwidth}
			
		\begin{overpic}[width=0.95\textwidth]{../figures/check-ripped.png}
		\put (0,140) { \parbox{0.8\textwidth}{
			\tiny
		\begin{itemize}
			\item Execution reaches \textbf{main}. \\Since the program was invoked as ``ccl -O fail.i'' variable \textbf{$argv[2]$} is now \textbf{``fail.i''}
			\item Execution reaches \textbf{combine\_instructions}. \\ Since $argv[2]$ was ``fail.i'', variable \textbf{$*first\_loop\_store\_insn\cecarrow fld[1].rtx \cecarrow fld[1].rtx \cecarrow fld[3].rtx \cecarrow fld[1].rtx$} is now \textbf{$\langle new rtx\_def\rangle$}.
			\item Execution reaches \textbf{if\_then\_else\_cond (95th hit)}. \\ Since $*first\_loop\_store\_insn\cecarrow fld[1].rtx \cecarrow fld[1].rtx \cecarrow fld[3].rtx \cecarrow fld[1].rtx$ was $\langle new rtx\_def\rangle$, variable \textbf{$link \cecarrow fld[0].rtx \cecarrow fld[0].rtx$} is now \textbf{$link$}.
			\item Execution ends. \\ Since variable $link \cecarrow fld[0].rtx \cecarrow fld[0].rtx$ was $link$, the program now \textbf{terminates with a SIGSEGV signal}. The program fails.
		\end{itemize}
		}}
		\end{overpic}

	
		\end{column}
	\begin{column}{0.6\textwidth}

		\begin{itemize}
			\item Zeller's prototypical \textit{HOWCOME} algorithm uses DD on program states as described above.
			\item When applied to \texttt{fail.c} the Cause-Effect-Chain to the left is returned.
		\end{itemize}		
		\phantom{ph}
		\bigskip
		\phantom{ph}
	\end{column}
	\end{columns}
	
\end{frame}

\begin{frame}{\textit{HOWCOME} on \texttt{fail.c}}
	\begin{columns}
		\begin{column}{.5\textwidth}
			\begin{overpic}[width=\textwidth]{../figures/check-end.png}
		\put (0,105) { \parbox{0.9\textwidth}{

			\small
		\begin{itemize}
			\item Execution reaches \textbf{if\_then\_else\_cond (95th hit)}. \\ Since $*first\_loop\_store\_insn\cecarrow fld[1].rtx \cecarrow fld[1].rtx \cecarrow fld[3].rtx \cecarrow fld[1].rtx$ was $\langle new rtx\_def\rangle$, variable \only<1,3>{\textbf{$link \cecarrow fld[0].rtx \cecarrow fld[0].rtx$} is now \textbf{$link$}} \only<2>{\red{\textbf{$link \cecarrow fld[0].rtx \cecarrow fld[0].rtx$} is now \textbf{$link$}}}.
			\item Execution ends. Since variable \only<1-2>{$link \cecarrow fld[0].rtx \cecarrow fld[0].rtx$ was $link$, the program now \textbf{terminates with a SIGSEGV signal}}\only<3>{\red{$link \cecarrow fld[0].rtx \cecarrow fld[0].rtx$ was $link$, the program now \textbf{terminates with a SIGSEGV signal}}}. The program fails.
		\end{itemize}
		}}
		\end{overpic}
		\end{column}
		
		\begin{column}{.5\textwidth}
			It is now easy to see for the developer why \texttt{fail.c} crashes GCC: \\[1em]

			\onslide<2->{\begin{enumerate}
				\only<2>{\item A \red{cycle in the RTL tree} of GCC is created at the 95th hit of \textbf{if\_then\_else\_cond}.}
				\onslide<3>{\item A cycle in the RTL tree of GCC is created at the 95th hit of \textbf{if\_then\_else\_cond}.} 
				\onslide<3>{\item The \red{cycle produces the error}.}
			\end{enumerate}}
		\end{column}
	\end{columns}
\end{frame}