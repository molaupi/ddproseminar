\subsection{The dd+ algorithm}

\newcommand{\ddp}{$dd^{+}$}

\begin{frame}{The \ddp\ algorithm}
	Zellers \textbf{extended} $dd$ algorithm \ddp\ deals with the following cases:
	\begin{enumerate}
		\item ``found (in $c_i$)'' \only<2->{$\Rightarrow$ continue search in $c_i$}
		\item ``interference'' \only<3->{$\Rightarrow$ search in $c_i$ while leaving $\bar{c_i}$ applied and vice versa}
		\item ``preference'' \only<4->{$\Rightarrow$ search in $c_i$ while leaving $\bar{c_i}$ applied}
		\item ``try again'' \only<5->{$\Rightarrow$ repeat search on same change set with twice as many subsets}
		%\item ``nothing left		
	\end{enumerate}
\end{frame}

\begin{frame}{The \ddp\ algorithm}
	Properties of \ddp:
	\begin{itemize}
		\item \ddp finds a minimal failure-inducing subset of changes as long as the changes are safe, i.e. combined they result in a consistent configuration.
		\item \ddp has at most linear time complexity like $dd$.
	\end{itemize}
\end{frame}

\begin{frame}{Use case for \ddp\: Analyzing input that crashes a program}
	Consider the C program $fail.c$ that crashed GCC in version 2.95.2:

	\begin{columns}
		\begin{column}{0.475\textwidth}
			\begin{align*}
	     &\text{double \textbf{mult}(double z[], int n)}\\
	     &\{\\
	     	&\quad \text{int } i,j; \\
	     	&\quad i = 0; \\
	     	&\quad for(j = 0; j < n; j++)\{ \\
	     	&\quad 	\quad i = i + j + 1; \\
	     &	\quad 	\quad z[i] = z[i] * (z[0] + 1.0); \\
	     &	\quad \} \\
	     &	\quad \text{return } z[n]; \\
	    & \}
	\end{align*} 
		\end{column}
		\begin{column}{0.475\textwidth}
			
		\end{column}
		
	\end{columns}

	
	
\end{frame}

\begin{frame}{How \ddp\ finds the minimal set of C tokens that causes the crash}
	\begin{tabular}{c l c}
				\hashtag & GCC input & test \\
				\hline
				1 & double mult($\dots$) $\{ int \; i, j; \; i = 0; \; for (\dots) \{ \dots\} \dots\}$ & $\xmark$ \\
				2 &\gray{ double mult($\dots$) $\{ int \; i, j; \; i = 0; \; for (\dots) \{ \dots\} \dots\}$} & $\cmark$ \\
				3 & double mult($\dots$) $\{\gray{ int \; i, j; \; i = 0; \; for (\dots) \{ \dots\} \dots}\}$ & $\cmark$ \\
				4 & double mult($\dots$) $\{ int \; i, j; \; i = 0; \; \gray{for (\dots) \{ \dots\} \dots}\}$ & $\cmark$ \\
				5 & double mult($\dots$) $\{ int \; i, j; \; i = 0; \; for (\dots) \{ \dots\}\gray{ \dots}\}$ & $\xmark$ \\
				6 & double mult($\dots$) $\{ int \; i, j; \; i = 0; \; for (\gray{\dots}) \{ \dots\} \gray{\dots}\}$ & $\cmark$ \\

				\vdots & \multicolumn{1}{c}{\vdots} & \vdots \\
				18 & \multicolumn{1}{c}{\dots \quad $z[i] = z[i] * (z[0] + 1.0);$ \quad \dots } & \xmark \\
				19 & \multicolumn{1}{c}{\dots \quad $z[i] = z[i] * (z[0] \gray{+ 1.0});$ \quad \dots } & \cmark \\
				20 & \multicolumn{1}{c}{\dots \quad $z[i] = z[i] * (z[0] \gray{+} 1.0);$ \quad \dots } & \qmark \\
				21 & \multicolumn{1}{c}{\dots \quad $z[i] = z[i] * (z[0] + \gray{1.0});$ \quad \dots } & \qmark \\
		\end{tabular}

\end{frame}